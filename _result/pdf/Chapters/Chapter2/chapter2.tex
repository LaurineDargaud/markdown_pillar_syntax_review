% -*- mode: latex; -*- mustache tags:  
\documentclass[10pt,twoside,english]{_support/latex/sbabook/sbabook}
\let\wholebook=\relax

\usepackage{import}
\subimport{_support/latex/}{common.tex}

%=================================================================
% Debug packages for page layout and overfull lines
% Remove the showtrims document option before printing
\ifshowtrims
  \usepackage{showframe}
  \usepackage[color=magenta,width=5mm]{_support/latex/overcolored}
\fi


% =================================================================
\title{Mardown and Pillar: syntax review}
\author{Laurine Dargaud}
\series{Pillar support}

\hypersetup{
  pdftitle = {Mardown and Pillar: syntax review},
  pdfauthor = {Laurine Dargaud},
  pdfkeywords = {Markdown, Pillar, Pharo, Smalltalk}
}


% =================================================================
\begin{document}

% Title page and colophon on verso
\maketitle
\pagestyle{titlingpage}
\thispagestyle{titlingpage} % \pagestyle does not work on the first one…

\cleartoverso
{\small

  Copyright 2017 by Laurine Dargaud.

  The contents of this book are protected under the Creative Commons
  Attribution-ShareAlike 3.0 Unported license.

  You are \textbf{free}:
  \begin{itemize}
  \item to \textbf{Share}: to copy, distribute and transmit the work,
  \item to \textbf{Remix}: to adapt the work,
  \end{itemize}

  Under the following conditions:
  \begin{description}
  \item[Attribution.] You must attribute the work in the manner specified by the
    author or licensor (but not in any way that suggests that they endorse you
    or your use of the work).
  \item[Share Alike.] If you alter, transform, or build upon this work, you may
    distribute the resulting work only under the same, similar or a compatible
    license.
  \end{description}

  For any reuse or distribution, you must make clear to others the
  license terms of this work. The best way to do this is with a link to
  this web page: \\
  \url{http://creativecommons.org/licenses/by-sa/3.0/}

  Any of the above conditions can be waived if you get permission from
  the copyright holder. Nothing in this license impairs or restricts the
  author's moral rights.

  \begin{center}
    \includegraphics[width=0.2\textwidth]{_support/latex/sbabook/CreativeCommons-BY-SA.pdf}
  \end{center}

  Your fair dealing and other rights are in no way affected by the
  above. This is a human-readable summary of the Legal Code (the full
  license): \\
  \url{http://creativecommons.org/licenses/by-sa/3.0/legalcode}

  \vfill

  % Publication info would go here (publisher, ISBN, cover design…)
  Layout and typography based on the \textcode{sbabook} \LaTeX{} class by Damien
  Pollet.
}


\frontmatter
\pagestyle{plain}

\tableofcontents*
\clearpage\listoffigures

\mainmatter

\chapter{Equivalence of notations}\section{Headers for sections}
In Markdown: \textcode{\#}

\begin{displaycode}{plain}
 # Title 1
## Title 2
### Title 3
\end{displaycode}

In Pillar: \textcode{!}

\begin{displaycode}{plain}
! Title 1
!! Title 2
!!! Title 3
\end{displaycode}
\section{Bulleted lists}
In Markdown: \textcode{*}

\begin{displaycode}{plain}
* element a
* element b
* element c
\end{displaycode}

In Pillar: \textcode{-}

\begin{displaycode}{plain}
- element a
- element b
- element c
\end{displaycode}
\section{Numbered lists}
In Markdown: n.

\begin{displaycode}{plain}
1. first element
2. second element
3. third element
\end{displaycode}

In Pillar: \textcode{\#}

\begin{displaycode}{plain}
# first element
# second element
# third element
\end{displaycode}
\section{Nested lists}
In Mardown: just a tab

\begin{displaycode}{plain}
1. chap1
    1. chap 1-1
    2. chap 1-2
        - element a 
        - element b
2. chap 2
\end{displaycode}

In Pillar: write associated symbols after one another

\begin{displaycode}{plain}
# chap 1
## chap 1-1
## chap 1-2
##- element a 
##- element b
# chap 2
\end{displaycode}

It gives:

\begin{enumerate}
\item chap 1
\begin{enumerate}
\item chap 1-1
\item chap 1-2
\begin{itemize}
\item element a 
\item element b
\end{itemize}

\end{enumerate}

\item chap 2
\end{enumerate}
\section{Formatting}
You surround the text you want to format with the following symbol:

\begin{itemize}
\item \textbf{bold}: \textcode{**your text**} in Markdown, \textcode{\symbol{34}\symbol{34}your text\symbol{34}\symbol{34}} in Pillar
\item \textit{italic}: \textcode{*your text*} in Markdown, \textcode{''your text''} in Pillar
\item \textcode{code}: \textcode{`your text`} in Markdown, \textcode{==your text==} in Pillar
\end{itemize}
\section{Links}
In Markdown:

\begin{displaycode}{plain}
[name_of_link](url)
\end{displaycode}

In Pillar:

\begin{displaycode}{plain}
*name_of_link>url*
\end{displaycode}
\section{Images}
In Markdown:

\begin{displaycode}{plain}
![name_of_picture](link_to_picture)
\end{displaycode}

In Pillar:

\begin{displaycode}{plain}
+name_of_picture>link_to_picture+
\end{displaycode}
\section{Code blocks}
In Markdown: a new line, then a tab and a new line at the end of your code

\begin{displaycode}{plain}
/ a new line
    type your code with a tab before
/ a new line
\end{displaycode}

In Pillar:

\begin{verbatim}
[[[
type your code
]]]
\end{verbatim}
\section{Notes \& Quotes}
Note and quote are not exactly equivalent, but both are used to insert a text in another context.

In Markdown:

\begin{displaycode}{plain}
> your quote
\end{displaycode}

In Pillar: use an anchor

\begin{displaycode}{plain}
@@note your note
\end{displaycode}



\bibliographystyle{alpha}
\bibliography{book.bib}

% lulu requires an empty page at the end. That's why I'm using
% \backmatter here.
\backmatter

% Index would go here

\end{document}
